%! Author = plumpie
%! Date = 3/29/20

% Preamble
\documentclass[../main.tex]{subfiles}

% Packages
%\usepackage{amsmath}
%\usepackage[
%    sorting=nyt,
%    bibstyle=gost-authoryear,
%    citestyle=gost-authoryear
%    backend = bibtex,
%    style = numeric
%]{biblatex}

%\addbibresource{main.bib}
%\bibliography{main}
%\bibliographystyle{plain}
%\usepackage[T2A, T1]{fontenc}
%\usepackage[english]{babel}
%\usepackage[russian]{babel}
% \usepackage{csquotes}


%\bibliography{main}
%\bibliographystyle{plain}





% Document
\begin{document}

    % \section*{CoverPage}
    %     \begin{itemize}
    %         \item Lalalala
    %         \item Наименование вуза, факультета, кафедры
    %         \item Тема работы на русском и английском языке
    %         \item Имя, фамилия, номер группы автора работы
    %         \item Должность, ученая степень, фамилия,
    %         инициалы научного руководителя ВКР
    %         \item место и год написания
    %     \end{itemize}
	
    \begin{abstract}
        And more, and more, and more

        Write something
Anfel
        Let me tell you about the Riemann-Hilbert problem on elliptic curve.
        Let us take a meromorphic matrix function $A(z)$ called \textit{the matrix of coeffitients} over a
        complex manifold and solve the equation
        $\dot{y} = A(z)y$. One will get a linear space of solutions, but the
        solutions have a monodromy map associated with singular points of the coefficients, which can be represented as a set of linear
        maps on the space of solutions.
        Each monodromy map correspodns to a specific singular point of $A$.
        Taking these maps together, a representation of
        the fundamental group of punctured surface is constructed.
        One assigns every loop class to the corresponding monodromy
        of the equation.
        But, given the monodromy representation, how can one find $A$ which corresponds to this representation?
        This question is a naive statement of a Riemann-Hilbert problem,
        or Inverse Monodromy problem.

        It turns out that one can generalize the above construction and treat the solutions
        as the sections of a semistable zero-degree bundle and $A$ as the matrix of a flat connection on the bundle.
        In this more general setting lies a huge opportuity, because
        the language of bundles and connections is inherently algebraic, contrary to the analytic one
        which was presented in the beginning.
        I confine the generalization to the case of semistable bundles since their Chern
        classes behave better than for a non-stable one, and
         if the bundle has suficiently large degree (the degree of the divisor
        of the arbitrary section), then the bundle has a set of global sections
        which form a basis at every point.
        The profit of this confinement will not be seen in the current inquisitions, but this
        property gives much in terms of algebraic structure of the bundle.
        Let us confine the considerations to the case of logarithmic (Fuchsian)
        connections.
        These are connections which have simple poles only.
       One can try to construct the connections and solutions explicitly.
        
        I want to construct such connections in
        a two-dimensional exceptional (non-splitting)  (see Thm5 in Atyiah's paper \cite{atiyah1957vector})
        bundle over an elliptic curve.
        I expect to get the explicit formulas for the connections, and find which monodromies they represent.
%        \begin{itemize}
%            \item Предполагаемые результаты исследования
%            \item Инструменты исследования
%            \item Наиболее интересные связи с раннее ивестными результатами
%        \end{itemize}


    \end{abstract}

    

    \section*{Preliminaries} \label{sec:prelim}
    Определения и теоремы, которые входят в Introduction
    %     Понятие когерентного пучка обобщает понятие локально
    %   свободного пучка конечного ранга.
    %     Локальносвободный пучок конечного ранга на многообразии
    % в точности соответствует векторному расслоению.
    Далее, про постановку задачи есть небольшая проблема.
    Теоретически она из двух частей.
Первая это выяснить как вообще устроены расслоения на эллиптической кривой,:
    сначала одномерные а потом двумерные, как их удобно описывать,
    как устроены логарифмические связности в них,
    как они пишутся в явном виде через тэта-функции,
    причём здесь вообще тэта-функции.

    Let me first provide a couple of examples. 
    Let us condider a fundamental matrix, and
    construct the equation corresponding
    to this fundamental matrix.
        \begin{equation}
                :
        \end{equation}
    I will assume that the definition of the vector bundle is known.

    Вторая часть - это после освоения первой части порешать какую-нибудь практическую задачу.
    Например, найти общий вид связности в исключительном двумерном расслоении
    или описать калибровочные симметрии связности в полустабильном,
    или найти аналог формулы с экспонентой вычета для монодромий по а- и б- циклам.
    % И вроде бы первая часть у нас более-менее случилась,
    %что уже, конечно, весьма немало, а до второй мы как-то пока не дошли.


    \subsection*{Methods of computations}\label{subsec:методы-явных-вычислений}


    \section*{Introduction}\label{sec:introduction}
        % \begin{itemize}
        %     \item Постановка задачи ВКР (Понятно математику)
        %     \item Мотивировка
        %     \item История вопроса
        % \end{itemize}




    Let us have an elliptic curve which is a holomorphic one-dimensional
    manifold of genus $1$.
%        В этой работе исследуются плоские логарифмические связности
    In my bachelor thesis I will treat flat logarighmic connections
%        в голоморфных расслоениях размерности 1 и 2
    in holomorphic bundles of rank $2$ and degree $0$ over elliptic curve.
%        степени 0 над эллиптической кривой.
    The curve has a fixed modular parameter $\tau$.
    The bundle will be called $E$ and the base will be called $B$.
%        Эллиптическая кривая фиксирована с точностью
%        до голоморфного диффеоморфизма.
%        В работе явно опиcываются такие связности в частных
%        случаях, с подробными вычислениями.
%        Мы решаем вопрос, как выглядят данные связности,
%        и какие ограничения на набор координатных выражений таких
%        связностей существуют.
%
    Horizontal sections of the connection are such sections that
    $\nabla s = 0$ or $d s = A dz \otimes s $ holds where both sides belong to
    $E \otimes T^*M$. The matrix $A$ is called the \textit{connection matrix} in this setting. 
    Any  meromorphic matrix that is changed along the bundle sections
    such that the horizontal section stays horizontal with
    respect to this modified matrix is a
    connection matrix for a particular bundle.
    Matveeva and Poberezhny in their article \cite{matveeva2017two} consider the case of a connection over the stable direct sum of line bundles.
    % They proved that the answer always exist.
    They find a Fuchsian connection corresponding to a rigid representation
    (here: the representation induced from a sphere) with $3$ branch points.
    They find explicitly a conection which is associated to the stable bundle
    $\mathcal{O}_{\lambda} (0) \oplus \mathcal{O}_{-\lambda} (0)$ for
    given $\lambda$.
    The logarithm of a local monodromy matrix of the found connection is conjugated to
    the residue of the matrix of the connection at the given monodromy point.
    Then, using these facts they find the solution for local monodromies and
    give an integral equation that should be
    satisfied by a conjugated (deformed) connection matrix so the global
    monodromy was trivial. See Preliminaries for additional

    There is a couple of thoughts I have not yet elaborated.
    First, it would be marvelous to infer some general properties
    of those connections
%        В то же время предпринимается попытка применить
%        базовые факты теории когомологий, такие как
    using Serre Duality, Riemann-Roch cohomology theorem, Chern characters
%        двойственность Серра, теорема Римана-Роха и гомологической
    and $Ext$ groups, which are standard algebraic
    instruments when one is studying complex manifolds.
%        алгебры, такие, как производные функторы, для
%        понимания структуры этих пространств.
    Second, I like to think about connections as
%        Вторым методом исследования является рассмотрение
%        связности как расслоения,
        sections itself, since they are actually special elements of
        $F = Hom(E, E \otimes T^*M )$ which is a bundle itself.
        The Leibniz rule and the flattness are two conditions which I
    should restate in this settings so I could distinguish \enquote{good}
    sections from \enquote{bad} ones.
    The Fuchsianness of the connection means that there are no divisor points
    with multiplicity more than $1$.
%        построенного по исходному расслоению.
%        Перспективным представляется случай стабильных
%        и полустабильных расслоений и стабильных
%        и полустабильных связностей, потому что свойство стабильности
%        и полустабильности широко изучается в алгебраической геометрии.
%

    The classic analytic way to operate with the sections are
%        Известно, что тета-функции Якоби и тета-функции Аппеля
    Jacoby $\theta$-functions, as far as I know they
    were discovered in the beginning of XIX century.
%        позволяют выразить сечения расслоений ранга $1$ и $2$.
    Any line-bundle has sections, represented by a product of theta-functions.
%        Также классическим подходом к исследованию расслоений
%        является рассмаотрение модулярной кривой расслоений.
    I will heavily use it in our constructions.
    Another classic object is Hypergeometric Differential Equation.
    The canonical form is
    \[ z(1-z) \frac{d^2y}{dz^2} + [c-(a+b+1)z] \frac{dy}{dz} - aby\]
    This equation has regular singular points at $0,1, \infty$, and different
    monodromies depending on $a,b,c$.
    This second-order linear differential equation can be represented as
    \begin{equation}
        \begin{pmatrix}
            0 & 1 \\
            \frac{ab}{z(z-1)} & \frac{c - (a+b+1)z}{z(z-1)} 

        \end{pmatrix}  \begin{pmatrix} y_0 \\ y_1 
                           \end{pmatrix} = \frac{d}{dz} \begin{pmatrix} y_0 \\ y_1
                                               \end{pmatrix}
    \end{equation}
        this equation has a two-dimensional space of solution.
        The Riemann-Hilbert problem has a rich history, that will not be covered here.
        Riemann in 1857 has published a paper where he considered the hypergeometric
    equations and its monodromies.
        Hilbert in 1905 solved the regular Riemann-Hilbert problem
        for rank $2$ bundles.
%        Эллиптические кривые являются классическим и неисссякаемым объектом
%        исследований, начиная с трудов Абеля и Якоби начала XIX века.
%        Они являются первым "Нетривиальным примером", на котором
%        математики могут попробовать свои силы.
        The Riemann-Hilbert problem on the Riemann
        sphere was solved negatively by Andrey Bolybruch in $1988$,
        using the techniques leading to many other results. \cite{Ilyashenko08lectureson}
    He has provided an explicit counterexample of a punctured Riemann sphere fundamental
    group representation that can not be a monodromy
    representation of a Fuchsian system.
    Bolybruch has obtained other results, one of them is that for
    any \textit{irreducible} representation can be obtained as a monodromy.
%        Исследование расслоений
%        актуально, потому что расслоения и связности в них
%        являются объектом изучения алгебраической геометрии,
%        теории интегрируемых систем, дифференциальной геометрии,
%        теории дифференциальных уравнений.
    The Riemann-Hilbert problem has connections with integrable systems \cite{fokas1991} and
    algebraic geometry.

%        Мамфорд показал, что пространство модулей стабильных расслоений
%        заданного ранга и степени является квазипроективным
%        алгебраическим многообразием, и его
%        когомологии также известны (Хардер-Наразимхан, 1975).
%        В статье Нитсуре(1993) дается определение логарифмической
%        связности как когерентного пучка модулей дифференциальных форм с
%        логарифмическими особенностями, у которого отсутствует кручение.
%        В этой же статье в proposition 3.4 говорится, что существует
%        расслоение, что все полустабильные логарифмические
%        связности с данным полиномом Гильберта получаются как его
%        факторы.






    \section*{Main Results}\label{sec:main-results}
    %     Результаты~
    % Возможно, есть связь между когомологиями пучка, соответствующего связности,
    % и когомологиями пучка исходного расслоения.


%    \cite{book}

%    \printbibliography
%Sets the bibliography style to UNSRT and imports the 
%bibliography file "samples.bib".

%придется в итоге дописывать в Overleaf
\bibliographystyle{unsrt}
\bibliography{main}

    % \section{Appendices}\label{sec:appendices}



\end{document}
