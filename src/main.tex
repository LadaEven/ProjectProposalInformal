%! Author = plumpie
%! Date = 3/29/20

% Preamble
\documentclass[11pt]{article}

% Packages
\usepackage{amsmath}
%\usepackage[
%    sorting=nyt,
%    bibstyle=gost-authoryear,
%    citestyle=gost-authoryear
%    backend = bibtex,
%    style = numeric
%]{biblatex}

%\addbibresource{main.bib}
%\bibliography{main}
%\bibliographystyle{plain}
\usepackage[T2A, T1]{fontenc}
\usepackage[russian]{babel}
\usepackage{csquotes}


%\bibliography{main}
%\bibliographystyle{plain}

% Document
\begin{document}

    \section*{CoverPage}
        \begin{itemize}
            \item Lalalala
            \item Наименование вуза, факультета, кафедры
            \item Тема работы на русском и английском языке
            \item Имя, фамилия, номер группы автора работы
            \item Должность, ученая степень, фамилия,
            инициалы научного руководителя ВКР
            \item место и год написания
        \end{itemize}

    \begin{abstract}
        Let me tell about the Riemann-Hilbert problem on elliptic curve.
        We take a meromorphic matrix function $A(z)$ over a
        complex manifold and solve the equation
        $\dot{y} = A(z)y$ depending on $A$ we will get different solution, but the
        solutions has a monodromy, which can be represented as a set of linear
        maps.
        Each map correspods to a singular point of a $A$.
        This takes us to a monodromy, which is a representation of
        the fundamental group of punctured surface.
        But, given the monodromy, how can we find $A$ such that $A$ satisfies it?

        It turns out that we can generalize our construction and treat the solutions
        as a trivial bundle sections and $A$ as the matrix of a flat connection inside the bundle.
        In this more general setting a huge opportuity potentially slumbers, because
        the language of bundles and connections is more algebraic than the analytic one
        which was presented in the beginning.
        Let us confine the generalization to the case of stable bundles since their Chern
        classes behave better than for a non-stable one, and
         if the bundle has suficiently large degree. (The degree of the divisor
        of the arbitrary section), then the bundle has a set of global sections
        which form a basis at every point.
        The profit of this confinement will not be seen in out current inquisitions, but this
        property gives much in terms of algebraic structure of the bundle.
        However, one can construct the connection explicitly.

        \begin{itemize}
            \item Предполагаемые результаты исследования
            \item Инструменты исследования
            \item Наиболее интересные связи с раннее ивестными результатами
        \end{itemize}


    \end{abstract}


    \section{Introduction}\label{sec:introduction}
        \begin{itemize}
            \item Постановка задачи ВКР (Понятно математику)
            \item Мотивировка
            \item История вопроса
        \end{itemize}

        Let us have an elliptic curve which is a holomorphic manifold.
    There is a theorem in \[Matveeva-Poberezhny\] article which deals
    with the case of a connection over a two-dimensional stable bundle.
    They find a connection corresponding to a rigid representation
    (here: the representation induced from a sphere) with $3$ branch points.



    \section{FirstDraftIntroduction}\label{sec:firstdraftintroduction}
%        В этой работе исследуются плоские логарифмические связности
    The work treats flat logarighmic connections
%        в голоморфных расслоениях размерности 1 и 2
    in holomorphic bundles of rank $2$ and degree $0$ over elliptic curve.
%        степени 0 над эллиптической кривой.
    The curve has a fixed modular parameter $\tau$
%        Эллиптическая кривая фиксирована с точностью
%        до голоморфного диффеоморфизма.
%        В работе явно опиcываются такие связности в частных
%        случаях, с подробными вычислениями.
%        Мы решаем вопрос, как выглядят данные связности,
%        и какие ограничения на набор координатных выражений таких
%        связностей существуют.
%
    Horizontal sections of the connection are such sections that
    $\Nabla s = 0$ or $ $
    Any matrix function that changes along the bundle sections
    such that the horizontal section stay horizontal fits as a
    connection matrix for a particular bundle.
    It would be marvelous to infer some general properties
    of those connections
%        В то же время предпринимается попытка применить
%        базовые факты теории когомологий, такие как
    using Serre Duality, Riemann-Roch cohomology theorem
%        двойственность Серра, теорема Римана-Роха и гомологической
    and $Ext$ groups.
%        алгебры, такие, как производные функторы, для
%        понимания структуры этих пространств.

%        Вторым методом исследования является рассмотрение
%        связности как расслоения,
%        построенного по исходному расслоению.
%        Перспективным представляется случай стабильных
%        и полустабильных расслоений и стабильных
%        и полустабильных связностей, потому что свойство стабильности
%        и полустабильности широко изучается в алгебраической геометрии.
%
%        Известно, что тета-функции Якоби и тета-функции Аппеля
%        позволяют выразить сечения расслоений ранга $1$ и $2$.
%        Также классическим подходом к исследованию расслоений
%        является рассмаотрение модулярной кривой расслоений.
%        Эллиптические кривые являются классическим и неисссякаемым объектом
%        исследований, начиная с трудов Абеля и Якоби начала XIX века.
%        Они являются первым "Нетривиальным примером", на котором
%        математики могут попробовать свои силы.
%        Исследование расслоений
%        актуально, потому что расслоения и связности в них
%        являются объектом изучения алгебраической геометрии,
%        теории интегрируемых систем, дифференциальной геометрии,
%        теории дифференциальных уравнений.
%
%        Мамфорд показал, что пространство модулей стабильных расслоений
%        заданного ранга и степени является квазипроективным
%        алгебраическим многообразием, и его
%        когомологии также известны (Хардер-Наразимхан, 1975).
%        В статье Нитсуре(1993) дается определение логарифмической
%        связности как когерентного пучка модулей дифференциальных форм с
%        логарифмическими особенностями, у которого отсутствует кручение.
%        В этой же статье в proposition 3.4 говорится, что существует
%        расслоение, что все полустабильные логарифмические
%        связности с данным полиномом Гильберта получаются как его
%        факторы.



    \section{Preliminaries} \label{sec:prelim}
       Определения и теоремы, которые входят в Introduction
        Понятие когерентного пучка обобщает понятие локально
       свободного пучка конечного ранга.
        Локальносвободный пучок конечного ранга на многообразии
        в точности соответствует векторному расслоению.
    \subsection{Методы явных вычислений}\label{subsec:методы-явных-вычислений}



    \section{Main Results}\label{sec:main-results}
        Результаты~
    Возможно, есть связь между когомологиями пучка, соответствующего связности,
    и когомологиями пучка исходного расслоения.


%    \cite{book}

%    \printbibliography

    \section{Appendices}\label{sec:appendices}



\end{document}
